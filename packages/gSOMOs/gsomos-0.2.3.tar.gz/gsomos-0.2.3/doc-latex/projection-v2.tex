%% LyX 2.4.3 created this file.  For more info, see https://www.lyx.org/.
%% Do not edit unless you really know what you are doing.
\documentclass[11pt]{article}
\renewcommand{\familydefault}{\sfdefault}
\usepackage{textcomp}
\usepackage[utf8]{inputenc}
\usepackage{booktabs}
\usepackage{amsmath}
\usepackage{amssymb}
\usepackage[a4paper]{geometry}
\geometry{verbose,tmargin=2cm,bmargin=2cm,lmargin=2cm,rmargin=2cm}
\usepackage{rotfloat}

\makeatletter

%%%%%%%%%%%%%%%%%%%%%%%%%%%%%% LyX specific LaTeX commands.
%% Because html converters don't know tabularnewline
\providecommand{\tabularnewline}{\\}

\@ifundefined{date}{}{\date{}}
%%%%%%%%%%%%%%%%%%%%%%%%%%%%%% User specified LaTeX commands.
\usepackage{graphicx}
\usepackage[table]{xcolor}

\title{SOMO Identification via Projection of $\alpha$ Orbitals onto the $\beta$ Subspace}
\author{}

\makeatother

\begin{document}
\maketitle

\section*{Methodological Details: Projection of Occupied Alpha Orbitals onto
the Beta Orbital Space}

Given a Gaussian log file from an unrestricted DFT calculation, we
extract the molecular orbital (MO) coefficients for both alpha and
beta orbitals, along with the AO overlap matrix $S$. The analysis
focuses on evaluating how each occupied alpha orbital projects onto
the full space spanned by all beta orbitals, which includes both occupied
and virtual ones.

Let $\phi_{i}^{\alpha}\in\mathbb{R}^{1\times n_{\text{basis}}}$ be
the coefficient vector of the $i$-th occupied alpha orbital, and
let $\Phi^{\beta}\in\mathbb{R}^{N\times n_{\text{basis}}}$ be the
matrix of all beta orbitals stored row-wise, where $N=n_{\beta}$
is the total number of beta orbitals. The projection vector is computed
as: 
\[
\mathbf{v}_{i}=\phi_{i}^{\alpha}\cdot S\cdot(\Phi^{\beta})^{T}\in\mathbb{R}^{1\times N}
\]
The squared norm $\|\mathbf{v}_{i}\|^{2}$ gives the total overlap
of the alpha orbital with the beta space.

To differentiate between the contributions from occupied and virtual
beta orbitals, we split the projection: 
\begin{align*}
\mathbf{v}_{i}^{\text{occ}} & =\phi_{i}^{\alpha}\cdot S\cdot(\Phi_{\text{occ}}^{\beta})^{T}\\
\mathbf{v}_{i}^{\text{virt}} & =\phi_{i}^{\alpha}\cdot S\cdot(\Phi_{\text{virt}}^{\beta})^{T}
\end{align*}
We then compute:

$\|\mathbf{v}_{i}^{\text{occ}}\|^{2}$ = projection of $\phi_{i}^{\alpha}$
onto occupied beta orbitals (noted {\small\textbf{P\texttwosuperior{}
$\boldsymbol{\beta_{\mathrm{occ}}}$}} in Table 1)

$\|\mathbf{v}_{i}^{\text{virt}}\|^{2}$ = projection of $\phi_{i}^{\alpha}$
onto virtual beta orbitals (noted {\small\textbf{P\texttwosuperior{}
$\boldsymbol{\beta_{\mathrm{virt}}}$}} in Table 1)

The total projection norm is decomposed to analyze how concentrated
or spread the projection is across beta orbitals: 
\begin{itemize}
\item The three largest values among the squared projections $v_{ij}^{2}$
are summed to compute \textbf{``Top 1 contrib (\%)}'', \textbf{``Top
2 contrib (\%)}'' and \textbf{``Top 3 contrib (\%)}''. 
\item The \textbf{dominance ratio} is defined as the largest single squared
projection divided by the total projection norm: $\max_{j}v_{ij}^{2}/\|\mathbf{v}_{i}\|^{2}$. 
\item The ``\textbf{$\beta$ MOs \textgreater 20\%}'' column lists all
beta orbitals contributing more than the specified percentage to the
squared projection norm, along with their contribution in the format
$[j,p_{j}]$, where $j$ is the index (1-based) and $p_{j}$ the percentage
contribution. For the most important contribution, it is nothing else
than the \textbf{dominance ratio}. It provides a direct quantitative
decomposition of each alpha orbital onto the beta orbital basis. Each
entry explicitly identifies the beta orbital(s) that significantly
compose the corresponding alpha orbital, along with their respective
percentage contributions
\end{itemize}
An orbital is flagged as a \textbf{SOMO candidate} if its projection
onto the virtual beta space exceeds 0.5 and its projection onto the
occupied beta space is below 0.5: 
\[
\|\mathbf{v}_{i}^{\text{virt}}\|^{2}>0.5\quad\text{and}\quad\|\mathbf{v}_{i}^{\text{occ}}\|^{2}<0.5
\]


\section*{Projection Results for Formaldehyde (H$_{2}$CO)}

The following table presents simplified projection data of occupied
alpha orbitals onto beta orbitals for formaldehyde. Orbitals identified
as SOMO (Single Occupied Molecular Orbitals) indicate significant
projection onto virtual beta orbitals and negligible projection onto
occupied beta orbitals.

\begin{sidewaystable}[ph]
\centering{}{\small{}%
\begin{tabular}{ccccccccc}
\toprule 
{\small\textbf{Alpha MO }} & {\small\textbf{Occ }}{\small$\boldsymbol{\alpha}$}{\small\textbf{ }} & {\small\textbf{Energy (Ha) }} & {\small\textbf{P\texttwosuperior{} $\boldsymbol{\beta_{\mathrm{virt}}}$ }} & {\small\textbf{P\texttwosuperior{} $\boldsymbol{\beta_{\mathrm{occ}}}$ }} & {\small\textbf{$\boldsymbol{\beta}$ MO{*} }} & {\small\textbf{Occ $\boldsymbol{\beta}$ }} & {\small\textbf{SOMO? }} & \textbf{$\boldsymbol{\beta}$ MOs \textgreater 20\%}\tabularnewline
\midrule
\midrule 
{\small\rowcolor{yellow!30}9} & {\small O } & {\small -0.204 } & {\small 0.996 } & {\small 0.004 } & {\small 9 } & {\small V } & {\small Y } & 9: 96.2\%\tabularnewline
{\small 8} & {\small O } & {\small -0.367 } & {\small 0.895 } & {\small 0.105 } & {\small 8 } & {\small V } & {\small Y } & 8: 88.4\%\tabularnewline
{\small 7} & {\small O } & {\small -0.438 } & {\small 0.005 } & {\small 0.995 } & {\small 7 } & {\small O } & {\small N } & 7: 97.5\%\tabularnewline
{\small 6} & {\small O } & {\small -0.487 } & {\small 0.000 } & {\small 1.000 } & {\small 6 } & {\small O } & {\small N } & 6: 97.9\%\tabularnewline
{\small 5} & {\small O } & {\small -0.529 } & {\small 0.106 } & {\small 0.894 } & {\small 5 } & {\small O } & {\small N } & 5: 89.4\%\tabularnewline
{\small 4} & {\small O } & {\small -0.672 } & {\small 0.002 } & {\small 0.998 } & {\small 4 } & {\small O } & {\small N } & 4: 99.4\%\tabularnewline
{\small 3} & {\small O } & {\small -1.096 } & {\small 0.001 } & {\small 0.999 } & {\small 3 } & {\small O } & {\small N } & 3: 99.6\%\tabularnewline
{\small 2} & {\small O } & {\small -10.238 } & {\small 0.000 } & {\small 1.000 } & {\small 2 } & {\small O } & {\small N } & 2: 100.0\%\tabularnewline
{\small 1} & {\small O } & {\small -19.240 } & {\small 0.000 } & {\small 1.000 } & {\small 1 } & {\small O } & {\small N } & 1: 100.0\%\tabularnewline
\bottomrule
\end{tabular}}{\small\caption{Projection of alpha molecular orbitals onto beta space for formaldehyde
(H$_{2}$CO), highlighting SOMOs.}
\label{tab:projection_formaldehyde} }
\end{sidewaystable}

\end{document}
