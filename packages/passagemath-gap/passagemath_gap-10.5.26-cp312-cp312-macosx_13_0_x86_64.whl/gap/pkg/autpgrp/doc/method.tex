%%%%%%%%%%%%%%%%%%%%%%%%%%%%%%%%%%%%%%%%%%%%%%%%%%%%%%%%%%%%%%%%%%%%%%%%%
%%
%A  method.tex         AutPGrp documentation                 Bettina Eick
%A                                                         Eamonn O'Brien
%%

%%%%%%%%%%%%%%%%%%%%%%%%%%%%%%%%%%%%%%%%%%%%%%%%%%%%%%%%%%%%%%%%%%%%%%%%%
\Chapter{The automorphism group method}

The {\AutPGrp} package installs a method for `AutomorphismGroup' for a 
finite $p$-group (see also Section~"ref:Groups of Automorphisms" 
in the {\GAP} Reference Manual).

\> AutomorphismGroup( <G> )   M 

The input is a finite $p$-group <G>. If the filters `IsPGroup', 
`IsFinite' and `CanEasilyComputePcgs' are set and true for <G>, 
the method selection of {\GAP}~4 invokes this algorithm. 

The output of the method is an automorphism group, whose generators 
are given in `GroupHomomorphismByImages' format in terms of their action 
on the underlying group <G>. 

\indextt{SetInfoLevel}
\>`InfoAutGrp' V

This is a {\GAP} InfoClass (these are described in Chapter~"ref:Info
Functions" in the {\GAP} Reference Manual). By assigning an <info-level>
in the range 1 to 4 via

\){\kernttindent}SetInfoLevel(InfoAutGrp, <info-level>)

varying levels of information on the progress of 
the computation, will be obtained. 

\beginexample 
gap> LoadPackage("autpgrp", false);
true

gap> G := SmallGroup( 32, 15 );
<pc group of size 32 with 5 generators>

gap> SetInfoLevel( InfoAutGrp, 1 );

gap> AutomorphismGroup(G);
#I  step 1: 2^2 -- init automorphisms 
#I  step 2: 2^2 -- aut grp has size 2
#I  step 3: 2^1 -- aut grp has size 32
#I  final step: convert
<group of size 64 with 6 generators>
\endexample

The algorithm proceeds by induction down the lower $p$-central
series of <G> and the information corresponds 
to the steps of this induction. In the following example we observe
that the method also accepts permutation groups as input, provided
they satisfy the required filters.

\beginexample 
gap> G := DihedralGroup( IsPermGroup, 2^5 );;
gap> IsPGroup(G);
true
gap> CanEasilyComputePcgs(G);
true
gap> IsFinite(G);
true
gap> A := AutomorphismGroup(G);
#I  step 1: 2^2 -- init automorphisms 
#I  step 2: 2^1 -- aut grp has size 2
#I  step 3: 2^1 -- aut grp has size 8
#I  step 4: 2^1 -- aut grp has size 32
#I  final step: convert
<group of size 128 with 7 generators>
gap> A.1;
Pcgs([ ( 2,16)( 3,15)( 4,14)( 5,13)( 6,12)( 7,11)( 8,10), 
  ( 1, 2, 3, 4, 5, 6, 7, 8, 9,10,11,12,13,14,15,16), 
  ( 1, 3, 5, 7, 9,11,13,15)( 2, 4, 6, 8,10,12,14,16), 
  ( 1, 5, 9,13)( 2, 6,10,14)( 3, 7,11,15)( 4, 8,12,16), 
  ( 1, 9)( 2,10)( 3,11)( 4,12)( 5,13)( 6,14)( 7,15)( 8,16) ]) -> 
[ ( 1, 2)( 3,16)( 4,15)( 5,14)( 6,13)( 7,12)( 8,11)( 9,10), 
  ( 1, 2, 3, 4, 5, 6, 7, 8, 9,10,11,12,13,14,15,16), 
  ( 1, 3, 5, 7, 9,11,13,15)( 2, 4, 6, 8,10,12,14,16), 
  ( 1, 5, 9,13)( 2, 6,10,14)( 3, 7,11,15)( 4, 8,12,16), 
  ( 1, 9)( 2,10)( 3,11)( 4,12)( 5,13)( 6,14)( 7,15)( 8,16) ]
gap> Order(A.1);
16
\endexample

%%%%%%%%%%%%%%%%%%%%%%%%%%%%%%%%%%%%%%%%%%%%%%%%%%%%%%%%%%%%%%%%%%%%%%
